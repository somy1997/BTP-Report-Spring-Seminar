% Discussion - Discuss the issues that you have got in the system, an idea about how you are trying to solve those. 
% Chapter Template

\chapter{Distributed Algorithm} % Main chapter title

\label{Chapter 5} % Change X to a consecutive number; for referencing this chapter elsewhere, use \ref{ChapterX}

\lhead{Chapter 5. \emph{Distributed Algorithm}} % Change X to a consecutive number; this is for the header on each page - perhaps a shortened title

\section{Assumptions}

We take the following assumptions for our distributed system :

{
\singlespacing
\begin{itemize}
    \item Each user has a unique id
    \item Each user knows other users
    \item Each user is a node in itself
    \item More than 2/3rd of the users (or specifically the committee members) are honest (non-faulty and form the largest sub) in the system (non-faulty)
    \item New requests for bond coin keep coming in
    \item Each user has a unique public address known to other user while its active
    \item Asynchronous system (Network Faults can happen)
    \item Eventually synchronous system (Eventually all network faults would be resolved)
    \item Reliable connection
    \item Each function is executed atomically
\end{itemize}
}

\section{Abbreviations}

Following are the abbreviations used to denote some fields and classes :
{
\singlespacing
\begin{itemize}
    \item id = User's ID
	\item mt = Transaction Type
	\item sid = Sender's ID
	\item rid = Reciever's ID
	\item t = Timestamp, can also be the id of the last block seen by the user to avoid time conflicts
	\item cc = CashCoin
	\item bc = BondCoin
	\item pk = Public Key of a user
	\item sk = Secret Key / Private Key of a user
	\item val = Transaction amount value that a user wants to send to another user
	\item h = hash obtained on applying a verifiable random function on the message using sk
	\item p = proof for verifying the hash of a message using that user's pk
	\item pa = public address for committee to communicate with this node during consensus
	\item fb = First Block in which a user starts giving consensus
	\item lb = Last Block after which a user stops giving consensus
	\item cps = Cryptocurrency Participation Score
	\item k = The number of blocks till which a transaction is valid. If a transaction is more than 
\end{itemize}
}

\section{Message Types}

Following messages will be exchanged in the system :

{
\singlespacing
\begin{enumerate}
    \item Transfer : Message gossiped by the a node with id sid to send val amount of cashcoins to a node with id rid at timestamp t.
    \begin{verbatim}
Message Transfer
{
    mt
    sid
    rid
    val
    tc
    t
    h
    p
}
    \end{verbatim}
    \item RequestBC : Message requesting to convert cashcoin to bondcoin and become a member of the committee.
    \begin{verbatim}
Message RequestBC
{
    mt
    sid
    pa
    tc
    t
    h
    p
}
    \end{verbatim}
    \item RequestCC : Message requesting to convert bondcoin to cashcoin before maturity.
    \begin{verbatim}
Message RequestCC
{
    mt
    sid
    tc
    t
    h
    p
}
    \end{verbatim}
    \item ChangePK : Message requesting to change pk.
    \begin{verbatim}
Message ChangePK
{
    mt
    sid
    pk // New Public Key
    tc
    t
    h
    p
}
    \end{verbatim}
    \item Join : Message requesting to join the system as a new user. The new id would be mentioned in the next few blocks as confirmation.
    \begin{verbatim}
Message Join
{
    mt
    pk
    tc
    t
    h
    p
}
    \end{verbatim}
    \item BlockCommit : Message broadcasted to give info about a new block.
    \begin{verbatim}
Message BlockCommit
{
    mt  // type of message
    sid // sender id
    block // block list element
    h // hash
    p // proof
}
    \end{verbatim}
\end{enumerate}
}

\section{Variables stored at each node}

{
\singlespacing
\begin{enumerate}
    \item myid = The ID of the user. This is allocated by the system when the user joins for the first time. It is an unsigned number and 0 is reserved for nodes who haven't joined yet. This id is used to index a user's information in the database.
	\item mysk = Private/Secret Key of the user. This is the only private state of the node (same as in algorand) and is used to sign the messages sent by it.
	\item mypk = The public key of the user. This key is advertised by the node so that other nodes can use it to verify its messages, it is unique to a node.
	\item mypa = The public address of the user for other users to send message when it takes part in the consensus. This address is only required when the user requests for a BondCoin.
	\item myfb = The most recent block from which user start participating in consensus
	\item mylb = The most recent block after which user stop participating in consensus
	\item mytc = User's transaction count to distinguish between two of its transaction in the same block, alternative to timestamp.
	\item hblock = Hash of the newest block seen.
	\item nusr = Number of users in the system, used to give id to a new user.
	\item bcuncommitted = Blocks heard but not yet committed. It is a priority queue arranged with respect to bid.
	\item trnuncommitted = Transactions heard but not committed. Transactions are arranged according to last block seen t and no duplicates are stored. The transactions are only stored by prospective BondCoin owners.
	\item nbr = Nearest neighbours in the network, information given by the network. This information is used for Distributed Broadcast of messages.
	\item bcq = A queue of users who have the bond coins and are participating in the consensus or going to participate in the future. No user can be twice in this q at a time. users in the queue can be accessed using their ids. Its like a dictionary but like a priority queue arranged using lb or fb. Implemented using double ended queue.
    Each element of the queue has following definition.
\begin{verbatim}
QueueElement BCUser
{
    id // ID of the user
    pk // Public Key of the user
    pa // Public Address of the user
    fb // first Block giving consensus
    lb // last Block giving consensus
    active // 1 if user is participating in consensus
}
\end{verbatim}
    \item usrd = A dictionary of users which maps a user's id (key here) to all the information about that user.
The values in the dictionary would have following definition
\begin{verbatim}
DictionaryKey id
DictionaryValue User
{
    pk
    lastpa
    cc
    cps
    fb  // the first block from which it joined
}
\end{verbatim}
    \item bcl = A list of blocks which acts as a unified ledger. Each block is a collection of transactions. Each transaction contains the original request and the response given after the consensus. A transaction is included in a block only if it is successful.
Each transaction has the following definition.
\begin{verbatim}
ListElement Transaction
{
    treq // Transaction Request, JSON string
    tres // Transaction Response, JSON string
}
\end{verbatim}
Each transaction can be of following types :
    \begin{enumerate}
        \item Transfer
        \begin{verbatim}
treq = 
{
    mt = `Transfer'
    sid
    rid
    val
    tc
    t
    h
    p
}
tres = 
{
    valfinal // Value transferred to the reciever after 
deducting the transaction fees
}
        \end{verbatim}
        \item RequestBC
        \begin{verbatim}
treq = 
{
    mt = `RequestBC'
    sid
    pa
    tc
    t
    h
    p
}
tres = 
{
    fb
    lb
}
        \end{verbatim}
        \item RequestCC
        \begin{verbatim}
treq = 
{
    mt = `RequestCC'
    sid
    tc
    t
    h
    p
}
tres =
{
    lb
    valfinal // Value to be added to user's account after 
putting fine on it. Fine put on the user as transaction 
fees for converting bc to cc before maturity.
}    
        \end{verbatim}
        \item ChangePK
        \begin{verbatim}
treq = 
{
    mt = `ChangePK'
    sid
    pk // New Public Key
    tc
    t
    h
    p
}
tres =
{
    fb // ID of the block from which the new public key 
would be valid
}    
        \end{verbatim}
        \item Join
        \begin{verbatim}
treq = 
{
    mt = `Join'
    pk
    tc
    t
    h
    p
}
tres =
{
    id // id of the new user
    fb // ID of the block from which the user would be valid
}
        \end{verbatim}
    \end{enumerate}
Each block has the following definition.
\begin{verbatim}
ListElement block
{
    bid // BlockID
    bpid // BlockProposerID
    tlist // List of Transaction Objects
    memlist // List of ids of BondCoin holders who participated 
in this round
    trnfees // Fees earned by each BondCoin owner in this round
    retirelist // List of ids of BondCoin holders who retire 
after this round
    joinlist // List of ids who are given allocated new bondcoins
    nusr // No. of users in the system after accepting new join 
requests in this block
    hprev // hash of previous block
    hq // hash of bcq after applying changes of this block of 
transactions
}    
\end{verbatim}
\end{enumerate}
}

\section{Initialisation}
{
\singlespacing
\begin{enumerate}
    \item myid = 0 or id given by the system when joined the first time
	\item mysk = secret key chosen by me
	\item mypk = public key corresponding to mysk
	\item mypa = public address given by the network to which other users can send messages to communicate with me
	\item mytc = 0 or transaction count of the last transaction
	\item hblock = 0 or hash of the last block in bcl
	\item nusr = 0 or nusr in the last block in bcl
	\item bcq = empty or Queue of users mirrored from a bondcoin holder
	\item usrd = empty or Dictionary of users mirrored from a bondcoin holder
	\item bcl = empty or List of blocks mirrored from a bondcoin holder
	\item bcuncommitted = empty
	\item trnuncommitted = empty
	\item nbr = nearest neighbours in the network, information given by the network
\end{enumerate}
}

\section{Functions}

\subsection{Abstract Functions}
These functions are called by other functions and serve specific tasks.
\begin{enumerate}
    \item Prepare : This function is called before pushing or broadcasting any message. This function sets t, h, p in a message
    \begin{lstlisting}
Abstract Prepare (Message M)
{
    if M.mt in {Transfer, RequestBC, RequestCC, ChangePK, Join} :
        // Can use some mutex while assigning timestamp to avoid 2 transactions getting same timestamp. Instead using tc (transaction count) here. Use mutex while assigning the tc value and incrementing it.
        mytc++
        M.tc = mytc
        M.t = bcl[-1].bid
    M.h = M.p = 0
    (M.h, M.p) = VRF (M, mysk)
}
    \end{lstlisting}
    \item Push : This function sends the message to all the bcowners in the user's queue. Put flag as 1 to send to only active members and 2 to send to all members.
    \begin{lstlisting}
Abstract Push (Message M, flag)
{
    for usr in bcq :
        if flag == 1 and usr.active == 0 :
            break
        if usr.active == -1 :
            continue
        Send M to usr.pa
}
    \end{lstlisting}
    \item Broadcast : This function is used to broadcast a message to the network.
    \begin{lstlisting}
Abstract Broadcast (Message M)
{
    // Implementing distributed broadcast here
    for usr in nbr :
        Send M to usr
}
    \end{lstlisting}
    \item Same : This function is for comparing two messages
    \begin{lstlisting}
Abstract same (treq t1, treq t2)
{
    // Here, we can compare sid but its not generic for all kinds of messages and following four will be enough
    // Added timestamp to distinguish between 2 similar transactions sent in the same block

    // Compare transaction type
    if t1.mt != t2.mt :
        return false
    
    // Compare transaction count
    if t1.tc != t2.tc :
        return false

    // Compare block of transaction
    if t1.t != t2.t :
        return false
    
    // Compare hash
    if t1.h != t2.h :
        return false
    
    // Compare proof
    if t1.p != t2.p :
        return false

    // This means that it is the same transaction, return true
    return true
}
    \end{lstlisting}
    \item Response : This function returns if the transaction was successful.
    \begin{lstlisting}
Abstract Response (Message M)
{
    // Check till k blocks if the transaction has been committed
    for i = M.t + 1 to M.t + k :
        // Check in every TCommit time if the new block has been added to bcl, here TCommit is avg commit time of a block
        while bcl[-1].bid < i :
            wait(TCommit)
        
        // Check in every transaction of the block to find own transaction
        for trn in enumerate(bcl[i].tlist) :
            if same(trn.treq, M) :
                return (true, trn.tres)
    
    // No match means the transaction wasn't accepted
    return (false, null)
}
    \end{lstlisting}
    \item : validUser : This function checks if the user has joined the system ie. it is a valid user or not.
    \begin{lstlisting}
Abstract validUser()
{
    // Check if the user is joined or not
    if myid == -1 :
        raiseError (' User not yet joined in the system ')
}
    \end{lstlisting}
    \item checkBCOwner : This function checks if a usr is BCOwner, pass flag as 1 to search only in current members and flag as 2 to search in all users.
    \begin{lstlisting}
Abstract checkBCOwner(id, flag)
{
    for usr in bcq :
        if flag == 1 and usr.active == 0 :
            break
        if usr.active == -1 :
            continue
        if usr.id == id :
            return true
    return false
}
    \end{lstlisting}
    \item verify : This function verifies a message sent by a user.
    \begin{lstlisting}
Abstract verify(Message M)
{
    if M.mt == Join :
        return VRFVerify(M, M.pk, M.h, M.p) // VRFVerify takes care of hashing when M.h = M.p = 0
    return VRFVerify()
}
    \end{lstlisting}
    \item ExecuteBlock : This function executes the transaction of a new block. If flag is 0 means just verifying the block, if flag is 1 means executing it.
    \begin{lstlisting}
Abstract ExecuteBlock (block, bcq, usrd, flag)
{
    if flag == 0 :

        if block.bid != bcl[-1].bid + 1 :
            return false
        
        // Can also check for bpid and put a hash, proof accordingly here

        // Can also check for priority of block proposer

        // Can make the above two commented checks in the function collecting block proposals itself
        
        // Check if previous hash matches
        if hblock != block.hprev :
            return false

        <Check if memlist matches the current active users in bcq>

        bcq = copy(bcq)
        usrd = copy(usrd)
        input = output = 0
        // Can also check the different memberlist or remove them entirely except memlist

    // Remember to execute RequestCC after RequestBC, can give a pass thru tlist at the end for RequestCC
    for trn in block.tlist :
        if flag == 0 and verify(trn.treq) == false :
            return false
        switch(trn.treq.mt)
            case Transfer :
                usrd[trn.treq.sid].cc -= val
                usrd[trn.treq.rid].cc += trn.tres.valfinal
                if flag == 0 :
                    if usrd[trn.treq.sid].cc < 0 :
                        return false
                    input += val
                    output += trn.tres.valfinal
            
            case RequestBC :
                usrd[trn.treq.sid].cc -= 1
                bcu = newBCUser
                bcu.id = trn.treq.sid
                bcu.pk = usrd[bcu.id].pk
                bcu.pa = trn.treq.pa
                bcu.fb = trn.tres.fb
                bcu.lb = trn.tres.lb
                bcu.active = 0
                bcq.push(bcu)
                if flag == 0 :
                    if usrd[trn.treq.sid].cc < 0 :
                        return false
                    input += 1
                    output += 1
            
            case RequestCC :
                bcq[trn.treq.sid].active = -1
                usrd[trn.treq.sid].cc += trn.tres.valfinal
                if flag == 0 :
                    input += 1
                    output += trn.tres.valfinal
            
            case ChangePK :
                usrd[trn.treq.sid].pk = trn.treq.pk
            
            case Join :
                usrd[trn.tres.id] = new User
                usrd[trn.tres.id].pk = trn.treq.pk
                usrd[trn.tres.id].lastpa = 0
                usrd[trn.tres.id].cc = 0
                usrd[trn.tres.id].cps = thres // here thres is the threshold
                usrd[trn.tres.id].fb = trn.tres.fb
    
    if flag == 0 :
        output += len(memlist)*block.trnfees
        if input != output :
            return false
        
        if len(usrd.keys) != block.nusr :
            return false

    // Add fees to members account
    for id in memlist :
        usrd[id].cc += block.trnfees
    
    // Pop retired members from bcq
    while bcq != empty and bcq.head.lb == bcl[-1].bid :
        bcq.pop()
    
    if flag == 0 :
        hq = hash(bcq)
        if hq != block.hq :
            return false
        return true
    
    // Means flag = 1
    hblock = hash(block)
    nusr = block.nusr
    
}
    \end{lstlisting}
\end{enumerate}

\subsection{Threads}
These functions run independently and may not be atomic.
\begin{enumerate}
    \item Sentinel : This thread runs at the start and tries to get user updated with the latest blocks.
\begin{lstlisting}
Thread Sentinel ()
{
    // run at all times except stops only when user goes offline

    time.reset()
    while(true) :
        
        // Resync if its past resynchronisation period
        if time.gone() > tresync :
            < Explicitly get missing blocks from other users and put it in bcuncommitted and broadcast them too >
            time.reset()

        
        // Commit the uncommitted block if any
        while bcuncommitted != empty and bcuncommitted.head.bid == bcl[-1].bid + 1 : // Here, in case of a bcl empty would need to add another condition, instead either put a dummy block at the start of bcl with id 0 or replace bcl[-1].bid with len(bid), if going with the former can also get the genesis block as a response to the Join message.
            bcl.append(bcuncommitted.head)
            bcuncommitted.pop()
            ExecuteBlock(bcl[-1], bcq, usrd, 1)
        
        // Listen for new message
        M = Listen() 

        // Verify only if not a new user
        if bcl != empty :
            verify(M)

        // Commit Message
        if M.mt == BlockCommit :
            
            // Already updated
            if bcl[-1].bid >= M.block.bid :
                continue
            else :
                bcuncommitted.push(M.block)

                // Only broadcast further if in the network
                if myid != 0 :
                    M.sid = myid
                    Prepare(M)
                    Broadcast(M)

        else if M.mt in {Transfer, RequestBC, RequestCC, ChangePK, Join} :
            // If I am a member
            if bcl[-1].id < mylb :
                trnuncommitted.push(M) // push removes duplicates and puts in the correct position in the queue, it also removes a transaction if already committed using function Response, and removes those which are already expired

            // If I am not an active member ignore
            else :
                continue

                
}
\end{lstlisting}
    \item Participate : This thread runs for the time the user has to participate in the consensus.
\begin{lstlisting}
Thread Participate()
{
    // mylb takes care that I am a member, for voting will also check if an active member, in normal operation the thread itself will destroy
    while bcl[-1].bid < mylb :
        <remove transactions from truncommitted who are accepted and those for which this block is the last hope and still they are not valid>
        
        // Take part only if in next block we are an active bcowner
        if bcl[-1].bid + 1 >= myfb :
            if <Proof of stake/current block makes me a proposer> :
                <Validate the transactions seen so far>
                <Propose a block>
                <Put priority>
                <Push it to the current active members>
                <Add it to own proposal list>
            
            while <proposal time> :
                <pop a new proposal>
                <check the validity of proposal>
                <put it as this round's block if highest priority> 

            <vote for the highest proposal seen>

            while <voting time> :
                <pop a vote>
                <add the vote>
                <see if the votes for a particular block exceeds 2/3rds of the total votes>
                <if it does commit this block and broadcast>
                <break> 

        <wait till new block is committed>
        <if not committed within some time restart consensus>

    destroy truncommitted
    Stop this thread
}
\end{lstlisting}
\end{enumerate}

\subsection{Public Functions}
These functions can be called directly by the user or are called depending on the events triggered by actions of the user.
\begin{enumerate}
    \item Wakeup : This function is called when the node boots up.
    \begin{lstlisting}
Wakeup () :
    Start thread Sentinel()
    \end{lstlisting}
    \item Shutdown : This function is called when the node is shutting down.
    \begin{lstlisting}
Shutdown () :
    If myid in bcq :
        <Push RequestCC>
        <Wait till the request is accepted>
    Stop thread Sentinel()
    \end{lstlisting}
    \item Transfer : This function is called by the user when it wants to transfer money to another user.
    \begin{lstlisting}
Transfer (rid, val)
{
    // Check if the user is joined or not
    validUser()
    
    // Check if reciever is valid or not
    if rid > nusr :
        raiseError (' Reciever is not valid ')

    // Check if val is valid
    if val > usrd[myid].cc :
        raiseError (' Not enough cashcoins to send ')

    // Prepare the message for sending
    mtransfer = new Message
    mtransfer.mt = Transfer
    mtransfer.sid = myid
    mtransfer.rid = rid
    mtransfer.val = val

    // Set tc, t, h, p
    Prepare (mtransfer)

    // Send the message to all the bcowners including those who have not yet started
    Push (mtransfer, 2)

    // Get response for the transaction
    (success, tres) = Response (mtransfer)

    // As the user's own accounts will also be handled in a general way, no use of tres here

    // return if the transaction was successful
    return success
}
    \end{lstlisting}
    \item RequestBC : This function is called by the user when it wants to participate in the consensus
    \begin{lstlisting}
RequestBC ()
{
    // Check if the user is joined or not
    validUser()

    // Check if the user has enough cashcoins
    if 1 > usrd[myid].cc :
        raiseError (' Not enough cashcoins to convert to bondcoin ')

    // Prepare the message for sending
    mrequestBC = new Message
    mrequestBC.mt = RequestBC
    mrequestBC.sid = myid
    mrequestBC.pa = mypa

    // Set tc, t, h, p
    Prepare (mrequestBC)

    // Send the message to all the bcowners including those who have not yet started
    Push (mrequestBC, 2)

    // Get response for the transaction
    (success, tres) = Response (mrequestBC)

    if success :
        // Set myfb and mylb
        myfb = tres.fb
        mylb = tres.lb
        create tnuncommitted
        start thread Particpate()

    // return if the transaction was successful
    return success
}
    \end{lstlisting}
    \item RequestCC : This function is called by the user when it wants to convert its bondcoin before maturity.
    \begin{lstlisting}
RequestCC ()
{
    // Check if the user is joined or not
    validUser()

    // Check if the user is in bcq
    if not checkBCOwner(myid, 2) :
        raiseError (' User not in BCOwner queue ')

    // Check if the user holds bc now
    if (bcl[-1].bid+1) == lb :
        raiseError (' User in BCOwner queue but this is last block ')
    
    // Prepare the message for sending
    mrequestCC = new Message
    mrequestCC.mt = RequestCC
    mrequestCC.sid = myid
    
    // Set tc, t, h, p
    Prepare (mrequestCC)

    // Send the message to all the bcowners including those who have not yet started
    Push (mrequestCC, 2)

    // Get response for the transaction
    (success, tres) = Response (mrequestCC)

    if success :
        // Set mylb
        //myfb = tres.fb
        mylb = tres.lb
        destroy trnuncommitted // can also just clear it
        Stop Thread Participate()

    // return if the transaction was successful
    return success
}
    \end{lstlisting}
    \item ChangePK : This function is called by the user when it wants to update its public key.
    \begin{lstlisting}
ChangePK (newpk)
{
    // Check if the user is joined or not
    validUser()

    // Prepare the message for sending
    mchangePK = new Message
    mchangePK.mt = ChangePK
    mchangePK.sid = myid
    
    // Set tc, t, h, p
    Prepare (mchangePK)

    // Send the message to all the bcowners including those who have not yet started
    Push (mchangePK, 2)

    // Get response for the transaction
    (success, tres) = Response (mchangePK)

    if success :
        // Set mypk
        mypk = newpk

    // return if the transaction was successful
    return success
}
    \end{lstlisting}
    \item Join : Called by a user to join the system for the first time. The bcowners check if the public key provided by the user is unique or not. If it is the user is given a unique id after which the user can start using the system. On success, the user would start mirroring the previous blocks, queue of users and dictionary of users from any member of the system.
    \begin{lstlisting}
Join ()
{
    // Can check if the user already exists but it might happen that same client wants to join as multiple user so we allow it here.

    // Prepare the message for sending
    mjoin = new Message
    mjoin.mt = Join
    mjoin.pk = mypk
    
    // Set tc, t, h, p
    Prepare (mchangePK)

    // Send the message to all the bcowners including those who have not yet started
    Push (mjoin, 2)

    // Get response for the transaction
    (success, tres) = Response (mchangePK)

    if success :
        // Set myid
        myid = tres.id

    // return if the transaction was successful
    return success
}
    \end{lstlisting} 
\end{enumerate}